%%%%%% This file contains example entries for various inclusions and should be followed strictly %%%%%%%

%% Itemization

\begin{itemize}
    \item Item 1
    \item Item 2
    \item Item 3
\end{itemize}


%% Enumeration
\begin{enumerate}
    \item Item 1
    \item Item 2
    \item Item 3
\end{enumerate}


%% Figures
\begin{figure}[h]
    \centering
    \includegraphics[width=1\linewidth]{Figures/image.png}
    \caption{Descriptive caption}
    \label{fig:descriprive_fig_name}
\end{figure}


%% Simple table
\begin{table}
    \centering
    \caption{Descriptive caption.}
    \begin{tabular}{c | c | c}
        \hline
        $n$ data & Degree $(n-1)$ & $\max|f(x) - p_{n-1}(x)|$ \\
        \hline
        2  & 1  & $9.61 \times 10^{-1}$ \\ 
        3  & 2  & $6.46 \times 10^{-1}$ \\ 
        4  & 3  & $7.07 \times 10^{-1}$ \\ 
        5  & 4  & $4.38 \times 10^{-1}$ \\
        \hline
    \end{tabular}
    \label{tab:descriptive_table_name}
\end{table}

%% Fancy table
\begin{table}[htbp]
    \caption{Table caption}
    \begin{center}
        \begin{tabular}{|c|c|c|c|}
            \hline
            \textbf{Table}&\multicolumn{3}{|c|}{\textbf{Table Column Head}} \\
            \cline{2-4} 
            \textbf{Head} & \textbf{\textit{Table column subhead}}& \textbf{\textit{Subhead}}& \textbf{\textit{Subhead}} \\
            \hline
            copy& More table copy$^{\mathrm{a}}$& &  \\
            \hline
            \multicolumn{4}{l}{$^{\mathrm{a}}$Sample of a Table footnote if applicable.}
        \end{tabular}
        \label{tab1}
    \end{center}
\end{table}

%% Code (inline)
Here is inline code: \verb|variable = value| or \texttt{function\_name()}.

\begin{verbatim}
def example_function(x):
    """This is an example function"""
    return x**2 + 2*x + 1
\end{verbatim}

%% Code (block using lstlisting - requires \usepackage{listings})
\begin{lstlisting}[language=Python, caption={Example Python code}, label={lst:example}]
import numpy as np

def compute_values(n):
    x = np.linspace(0, 1, n)
    y = x**2
    return x, y
\end{lstlisting}


% Equations (inline)
The equation $E = mc^2$ is famous, and we can also write $\int_0^\infty e^{-x} dx = 1$.

%% Equations (displayed, unnumbered)
\[
    \nabla \times \mathbf{E} = -\frac{\partial \mathbf{B}}{\partial t}
\]

%% Equations (displayed, unnumbered)
$$
    \nabla \times \mathbf{E} = -\frac{\partial \mathbf{B}}{\partial t}
$$

%% Equations (displayed, numbered)
\begin{equation}
    \frac{d^2 x}{dt^2} + \omega^2 x = 0
    \label{eq:harmonic_oscillator}
\end{equation}


%% Matrices
\begin{equation}
    \mathbf{A} = \begin{pmatrix}
        a_{11} & a_{12} & a_{13} \\
        a_{21} & a_{22} & a_{23} \\
        a_{31} & a_{32} & a_{33}
    \end{pmatrix}
\end{equation}


%% Cross-references
\Cref{fig:example} shows the results.  % "Figure 1 shows..."
See \cref{tab:data,tab:results}.       % "see tables 1 and 2."
From \cref{eq:newton}, we derive...    % "From equation (3), we..."